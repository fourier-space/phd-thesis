\addcontentsline{toc}{chapter}{Front Matter}

%%%%-%%%%-%%%%-%%%%=%%%%-%%%%-%%%%-%%%%=%%%%-%%%%-%%%%-%%%%=%%%%-%%%%-%%%%-%%%%

\addsec{Declaration of Originality}
The material presented in this thesis is my own work, except where otherwise
referenced or acknowledged.
\\
—\textbf{A. Freddie Page}, January 2016.

\vspace{\fill}

\addsec{Copyright Declaration}
The copyright of this thesis rests with the author and is made available under a
Creative Commons Attribution Non-Commercial No Derivatives licence.
Researchers are free to copy, distribute or transmit the thesis on the condition
that they attribute it, that they do not use it for commercial purposes and that
they do not alter, transform or build upon it.
For any reuse or redistribution, researchers must make clear to others the
licence terms of this work.

%%%%-%%%%-%%%%-%%%%=%%%%-%%%%-%%%%-%%%%=%%%%-%%%%-%%%%-%%%%=%%%%-%%%%-%%%%-%%%%

\clearpage

\addsec{Abstract}
\small

By reducing the number of dimensions that light can propagate in from three to
two, control over the properties of propagation can be achieved.
The plasmonic modes of planar metal-dielectric heterostructures
will confine light in one dimension, enhancing the
electromagnetic fields within the structure.
This thesis focuses on two particular aspects of active
nanoplasmonics in planar systems, stopped light lasing and plasmons with gain in
nonequilibrium graphene.

For stopped-light lasing, a plasmonic waveguide mode is designed to have two
points of zero group velocity in a narrow frequency range, in order to increase
the local density of optical states that a gain medium can emit into.
The two stopped light points form a band of slow light that supports a wide
range of wavevectors, allowing localisation over a sub-wavelength gain medium
and providing the feedback required for lasing.
This results in a new type of laser that does not rely on predefined cavity
modes, in fact is cavity-free in \twod, dynamically forming its lasing mode
about a locally pumped region of carrier inversion.

Graphene, a single-atom thick semimetal, provides the ultimate miniaturisation
as a truly \twod material.
It is shown that graphene can support plasmons with gain, under realistic
conditions of collision loss, temperature, doping, and carrier relaxation via
amplified spontaneous emission.
This is made possible by developing a scheme to evaluate polarisabilities
for nonequilibrium carrier distributions, allowing the calculation of the exact
\rpa complex-frequency plasmon dispersion solutions.
The rates of spontaneous emission are calculated and are critically dependant on
the exact dispersion relation.
The instantaneous rates are found to be 5 times faster than previously reported
and, when coupled with phonons, lead to carrier relaxations on $100\fs$
timescales.
The polarisability and relaxation rates must form the basis of any active
graphene device, where electromagnetic energy is coupled to an evolving
electronic system.


\normalsize

%%%%-%%%%-%%%%-%%%%=%%%%-%%%%-%%%%-%%%%=%%%%-%%%%-%%%%-%%%%=%%%%-%%%%-%%%%-%%%%

\clearpage

\addsec{Acknowledgements}
I'd like to give thanks to the following people who have all, either directly or
indirectly, supported me during my \phd:

Firstly, Prof. Ortwin Hess for supervising the project and giving me the
opportunity to study in the group at Imperial College London;
My collaborators at Imperial:
Andreas Pusch, Fouad Ballout, Kosmas Tsakmakidis, Sebastian Wüstner, and
Tim Pickering.
Though I would like to particularly thank Joachim Hamm for mentoring me,
providing endless advice and guidance, and generally being great fun to work
with;
Friends in and around the group not mentioned above:
Anne-Julie, Doris, Fabian, James, John, Ollie, Peter, Peter, Stephanie,
Seungbum, Tristan, Zara, and Zeph;
House mates, first at Mayton Street:
Claire, Liz, Sarah, and Karla who definitely followed \emph{me} to Imperial.
Then at Beit Hall:
Andreas, Anne, Alex, Deborah, Elly, Ling, Lucia, Nathalie, Sam, Sarahs, and
Savvas;
Band mates in \emph{Filthy Strings}: Amanda, Edwin, Michael, Michael,
Norbert, and Sam;
And to my family: Julie, Ian, Sammy, June and John, Lucinda, and Russell and
Tom, for bearing with me.
Finally, a special thanks goes to everyone who proof-read sections of the
thesis:
Andreas, Doris, Joachim, and Nathalie.

%%% All got to fit on one page so fuck formatting!!

\addsec{Epigraph}
\begin{displayquote}
This little inquiry has fused.\\
It is now growing faster than the speed of bloody light.\\
It's not going to be something that we can see from space,\\
It's going to \emph{be} space!
\end{displayquote}
—\textbf{Stewart Pearson}, \emph{The Thick of It}

%%%%-%%%%-%%%%-%%%%=%%%%-%%%%-%%%%-%%%%=%%%%-%%%%-%%%%-%%%%=%%%%-%%%%-%%%%-%%%%

\clearpage

\addsec{Preface to the \emph{Doctor’s Edition}}
Since passing my viva voce defense and submitting the “final” copy of my thesis,
I have used the document as a work of reference as a research associate.
As such, I've found cause to update certain sections, either to add short
related results, in some cases to change convention uses, and indeed to correct
minor mistakes spotted since submission.

The larger edits have concerned the transfer matrix method in the \emph{Theory
and Methods} section.
Specifically making the eigenvalue equation more succinct, and putting
transverse electric and magnetic solutions on the same footing by
placing focus on in-plane electric fields.

The result is this document, which I refer to, with tongue-in-cheek, as the
\emph{Doctor's Edition}.
It has not undergone further examination, and the original edition remains
availible online\footnote{At
\href{https://spiral.imperial.ac.uk:8443/handle/10044/1/30760}{
https://spiral.imperial.ac.uk:8443/handle/10044/1/30760}}.

For this edition I would like to thank Nuttawut and Illya who spotted a number
of the mistakes.
And to give particular thanks to Prof. Anatoly Zayats and Prof. Stefan Maier for
taking the time to read my thesis and conduct my viva.

%%%%-%%%%-%%%%-%%%%=%%%%-%%%%-%%%%-%%%%=%%%%-%%%%-%%%%-%%%%=%%%%-%%%%-%%%%-%%%%

\clearpage
\addcontentsline{toc}{section}{Contents}
\tableofcontents

%%%%-%%%%-%%%%-%%%%=%%%%-%%%%-%%%%-%%%%=%%%%-%%%%-%%%%-%%%%=%%%%-%%%%-%%%%-%%%%

\clearpage
\listoffigures
\listoftables

%%%%-%%%%-%%%%-%%%%=%%%%-%%%%-%%%%-%%%%=%%%%-%%%%-%%%%-%%%%=%%%%-%%%%-%%%%-%%%%

\clearpage
\let\defaultchaptermarkformat=\chaptermarkformat
\renewcommand*{\chaptermarkformat}{}
\addsec{Abbreviations, Acronyms, and Initialisms}
\chaptermark{Abbreviations, Acronyms, and Initialisms}

\begin{longtable}{ l l l }
Abbr. & Abbreviation & First use \\ \hline
\twod & Two dimensional \df & \sec{intro} \\
\threefive & Groups three and five (semiconductor) \df & \sec{plasSLS} \\
\ar & Auger recombination \df & \sec{grIntro} \\
\ase & Amplified spontaneous emission \df & \sec{4lvl} \\
\cfpd & Complex frequency plasmon dispersion \df & \sec{4lvl} \\
\cfr & Complex frequency \df & \sec{cwcf}\\
\cpml & Convolutional perfectly matched layer \df & \sec{lasingDynamics} \\
\cwv & Complex wavevector \df & \sec{cwcf}\\
\ea & Evolutionary algorithm \df & \sec{EA} \\
\fdtd & Finite difference time domain \df & \sec{fdtd} \\
\fft & Fast Fourier transform \df & \sec{lasingMode} \\
\fgr & Fermi's golden rule \df & \sec{sponEmit} \\
\ΓO & Γ optical (phonon) \df & \sec{optPhonons} \\
\ito & Indium Tin Oxide \df & \sec{plasSLS} \\
\KA & K acoustic (phonon) \df & \sec{optPhonons} \\
\KO & K optical (phonon) \df & \sec{optPhonons} \\
\LO & Longitudinal optical (phonon) \df & \sec{optPhonons} \\
\mdf & Massless Dirac fermion \df & \sec{grIntro} \\
\ma & Metal-air \df & \sec{hybrid} \\
\mim & Metal-insulator-metal \df & \sec{hybrid} \\
\mima & Metal-insulator-metal-air \df & \sec{hybrid} \\
\npe & Non-equilibrium plasmon emission \df & \sec{npe} \\
\rpa & Random phase approximation \df & \sec{grIntro} \\
\sl & Stopped light \df & \sec{sllIntro} \\
\spp & Surface plasmon polariton \df & \sec{intro} \\
\TE & Transverse electric \df & \sec{TETM} \\
\TM & Transverse magnetic \df & \sec{TETM} \\
\tmd & Transition metal dichalcogenide \df & \sec{intro} \\
\tmm & Transfer matrix method \df & \sec{introTMM} \\
\TO & Transverse optical (phonon) \df & \sec{optPhonons} \\
\trarpes & time resolved, angle resolved photo-emission spectroscopy \df &
\sec{grIntro} \\
\zgv & Zero group velocity \df & \sec{plasSLS} \\
\end{longtable}

%%%%-%%%%-%%%%-%%%%=%%%%-%%%%-%%%%-%%%%=%%%%-%%%%-%%%%-%%%%=%%%%-%%%%-%%%%-%%%%

\clearpage
\addsec{List of Symbols}
\chaptermark{List of Symbols}

\small

\newsavebox\ltmcbox

\begin{multicols}{2}
\medskip

\setbox\ltmcbox\vbox{
\makeatletter\col@number\@ne
\begin{longtable}{l l}

$A^\pm$				&					Forward/backward amplitude \\
$\mathrm{a}$		&					Absorption\\

$\B$				&					Magnetic field \\

$\Cn$				&					Complex contour\\
$\c$				&					Vacuum speed of light \\

$\D$				&					Electric displacement field \\
$\Dpl$				&					Plasmon density of states \\

$\E$				&					Electric field \\
$\mathrm{e}$		&					Electron \\
$\mathrm{e}$		&					Emission\\
$\ex$				&					Unit vector in $x$ direction \\

$\f$			&					Fermi function \\

$G$				&					Polarisability aux. function \\
$\Gp$			&					Pol. derivative aux. function \\
$\Gpl$			&					Recombination rate density \\

$\H$				&					Magnetic H field \\
$\mathrm{h}$		&					Hole \\
$\ħ$				&					Dirac's constant \\

$\J$				&					Current density \\
$\Js$				&					Surface current density \\

$\k$				&					Fermion wavevector \\
$\kB$				&					Boltzmann's constant \\

$\M$				&					Magnetisation field \\
$\Mtr$				&					Transfer matrix\\
$m$					&					Carrier inversion parameter \\
$\mc$				&					Critical branching parameter \\

$N$					&					Emitter density \\
$\Ne$				&					Fermion number density \\
$\Ndot$				&					Fermion net generation rate \\
$\NdotPump$			&					Number density pump rate \\
$\NdotRel$			&					Number density relaxation rate \\
$\npar$				&					Refractive index \\
$\nβ$				&					Boson number \\
$\nβdot$			&					Boson net generation rate \\
$\nβdotrel$			&					Boson relaxation rate \\
$\nβeq$				&					Boson number \\

$\P$				&					Polarisation field \\
$\mathrm{pl}$		&					Plasmon \\
$\mathrm{ph}$		&					Phonon \\

$\Q$				&					\threed boson wavevector\\
$\q$				&					\twod boson wavevector \\
$\qc$				&					High wavevector cutoff \\

$\Rspon$			&					Plasmon emission rate \\
$\Rβ$				&					Carrier relaxation rate \\
$\rp$				&					Electrical pump rate \\
$\rβλ(\q)$			&					Boson spectral emission rate \\

$\S$				&					Poynting vector \\
$\Sβλ$				&					Energy relaxation rate \\

$\T$				&					Temperature \\
$\Tamb$				&					Ambient temperature \\
$\Tc$				&					Carrier temperature \\
$\t$				&					Time \\

$\Ue$				&					Fermion energy density \\
$\Udot$				&					Energy density net rate \\
$\UdotPump$			&					Energy density pump rate \\
$\UdotRel$			&					Energy density relaxation rate \\
$\ud$				&					Dispersion loss \\
$\ug$				&					Group loss \\
$\uTE$				&					Transfer matrix layer parameter \\

$\vb$				&					Band velocity \\
$\vd$				&					Dispersion velocity \\
$\vF$				&					Fermi velocity \\
$\vg$				&					Group velocity \\
$\vp$				&					Phase velocity \\

$\x$				&					Position vector \\

$\Zf$				&					Vacuum impedance \\
$\Zpar$				&					Wave impedance \\

$\αf$				&					Vacuum fine structure constant \\
$\αg$				&					Graphene fine structure const.\\

$\γabs$				&					Plasmon absorption rate \\
$\γL$				&					Drude-Lorentz width \\
$\γemit$			&					Plasmon stim. emit. rate \\
$\γp$				&					Drude loss \\
$\γpl$				&					Plasmon frequency imag. part \\
$\γplabs$			&					Plasmon absorption rate \\
$\γplemit$			&					Plasmon stim emission rate \\
$\γpump$			&					Pump width \\
$\Γple$				&					Fermion recombination rate \\

$\ΔN$				&					Inversion density \\
$\Δq$				&					Stopped light spatial bandwidth \\
$\Δt$				&					\textsc{Fdtd} timestep \\
$\Δx$				&					\textsc{Fdtd} cell size \\
$\Δω$				&					\textsc{Sl} frequency bandwidth \\

$\εpar$				&					Permittivity \\
$\εbar$				&					Average permittivity\\
$\εbg$				&					Background permittivity\\
$\εf$				&					Vacuum permittivity \\
$\εinf$				&					High frequency permittivity \\
$\εL$				&					Drude-Lorentz permittivity \\
$\εRPA(\q,\ω)$		&					\textsc{Rpa} dielectric function \\
$\εtens$			&					Permittivity tensor \\
$\έ$				&					Fermion energy \\
$\έtilM$			&					Fermion matsubara frequency \\

$\μpar$				&					Permeability \\
$\μ$				&					Chemical potential \\
$\μbar$				&					Scale energy \\
$\μe$,$\μh$			&					Electron fermi level \\
$\μf$				&					Vacuum permeability \\
$\μtens$			&					Permittivity tensor \\

$\Π$				&					Polarisability \\
$\Πe$				&					Electron partial polarisability \\

$\ρ$				&					Charge density \\
$\ρs$				&					Sheet charge density \\

$\Σabs$				&					Plasmon absorption polarisability \\
$\Σemit$			&					Plasmon stim. emit. polarisability \\
$\σe$				&					Emission cross-section \\
$\σs$				&					Sheet conductivity \\
$\σsinter$			&					Interband sheet conductivity \\

$\τpl$				&					Plasmon lifetime \\

$\χe$				&					Electric susceptability \\
$\χL$				&					Drude-Lorentz susceptability\\
$\χm$				&					Magnetic susceptability \\

$\ω$				&					Boson frequency \\
$\ωL$				&					Drude-Lorentz freqeuncy\\
$\ωp$				&					Drude-Lorentz plasma frequency \\
$\ωpl$				&					Plasmon frequency solution \\
$\ωpump$			&					Pump frequency \\
$\ωsp$				&					Surface plasmon frequency \\

\end{longtable}
\unskip
\unpenalty
\unpenalty}

\unvbox\ltmcbox

\medskip
\end{multicols}

\normalsize

Many quantities in this list can apply to multiple species, i.e. $\nβ$ is the
number of bosons.
Similar quantities, $\npl$, $n_\mathrm{ph}$ apply more specifically to plasmons
and phonons.
For brevity these have been omitted.
The same applies to fermions, electrons, and holes, etc.
Also for brevity, variants of variables such as Fourier transforms, vector
norms, and scaled versions of quantities (discussed in \sec{scale}),
are omitted.



%%%%-%%%%-%%%%-%%%%=%%%%-%%%%-%%%%-%%%%=%%%%-%%%%-%%%%-%%%%=%%%%-%%%%-%%%%-%%%%

\clearpage
\addsec{Publications}
\chaptermark{Publications}
This thesis contains work drawn from the following publications:
\begin{itemize}
  \item
   \emph{Completely Stopped and Dispersionless Light in Plasmonic Waveguides},\\
   K.~ L.~Tsakmakidis, T.~W.~Pickering, J.~M.~Hamm, \textbf{A.~F.~Page},
   and O.~ Hess,\\
   Phys.~Rev.~Lett., vol.~112, p.~167401, Apr.~2014.
  \item
   \emph{Cavity-free plasmonic nanolasing enabled by dispersionless stopped
   light},\\
   T.~W.~Pickering, J.~M.~Hamm, \textbf{A.~F.~Page}, S.~Wuestner, and O.~Hess,\\
   Nat.~Commun., vol.~5, p.~4972, Sept.~2014.
  \item
   \emph{Ultrafast dynamics of nanoplasmonic stopped-light lasing},\\
   S.~Wuestner, T.~W.~Pickering, J.~M.~Hamm, \textbf{A.~F.~Page}, A.~Pusch, and
   O.~Hess,\\
   Faraday Discuss., vol.~178, pp.~307–324,~2015.
  \item
   \emph{Nonequilibrium plasmons with gain in graphene},\\
   \textbf{A.~F.~Page}, F.~Ballout, O.~Hess, and J.~M.~Hamm,\\
   Phys.~Rev.~B,~vol.~91, p.~75404, Feb.~2015.
  \item
   \emph{Nonequilibrium plasmon emission drives ultrafast carrier relaxation
   dynamics in photoexcited graphene},\\
   J.~M.~Hamm, \textbf{A.~F.~Page}, J.~Bravo-Abad, F.~J.~Garcia-Vidal, and
   O.~Hess,\\
   Phys.~Rev.~B,~vol.~93, p.~41408(R), Jan.~2016.
\end{itemize}

%%%%-%%%%-%%%%-%%%%=%%%%-%%%%-%%%%-%%%%=%%%%-%%%%-%%%%-%%%%=%%%%-%%%%-%%%%-%%%%


\clearpage
\chaptermark{~}
~
\clearpage
\renewcommand*{\chaptermarkformat}{\defaultchaptermarkformat}
