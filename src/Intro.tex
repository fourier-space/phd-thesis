\chapter{Introduction} \label{sec:intro}

Control over the propagation and interactions of light at the smallest
lengthscales is the subject of the field of nano-optics.
Whereas semiconductor based electronics has enjoyed a miniaturisation, driven by
industry, down to nanometre scales; interactions with light on this scale are
weak \cite{Dionne2010}.
This is because photons can not be confined to small volumes due to the
diffraction limit, which sets the minimum wavelength of light to be inversely
proportional to its frequency.
Semiconductor bandgaps are typically in the infrared frequency range, limiting
photons of this frequency to wavelengths in the high hundreds of nanometres.

\emph{Plasmonics} offers a solution to this size mismatch
\cite{Zayats2003,Gramotnev2010}.
Light couples to the oscillations of surface charges on a
metal-insulator interface as a \emph{surface plasmon polariton} (\spp) mode.
There are two primary features of \spps which make them useful in
nano-optics applications; \spps break the diffraction limit, i.e. they have
wavevectors larger than what is allowed for a photon of their given frequency;
secondly they have large field enhancements at the material interface.
The combination of shorter wavelengths and higher field densities allows for the
interaction with the electronic systems of active media
\cite{MacDonald2009,Hess2012},
that is usually weak with photons.
This allows for applications including
biosensing \cite{Anker2008},
plasmonic modulators \cite{Cao2009},
photodetectors \cite{Tang2008},
Nonlinear effects \cite{Kauranen2012},
and quantum optics \cite{Jacob2011,Tame2013}.

The simplest geometry in which to study \spps is the planar stack,
that is slabs of material stacked in one dimension, leaving the other two
dimensions uniform, and therefore permitting planewave solutions.
In this way, the dynamics of waves in \twod are determined by the composition of
the structure in the perpendicular direction.
Planar structures have the advantage of ease of fabrication and mechanical
stability.
Nanoplasmonic layered structures have been shown to have applications as
perfect lenses \cite{Pendry2000,Zhang2005a},
behave with hyperbolic dispersion \cite{Poddubny2013},
and offer more exotic setups such as “analogue computing” \cite{Silva2014}.

Planar structures provide a starting point for more complicated geometries,
these may be accessed by transforming the solutions using transformation optics
\cite{Pendry2006,Leonhardt2006},
or composing together multiple layered structures into nanopatterned
arrays \cite{Radko2009,Davies2013}, or devices \cite{Dionne2009}.

Layers in planar structures can be taken to the limiting case of
\emph{two-dimensional} (\twod) materials \cite{Novoselov2005a}.
These are molecularly thin layers, with heights of only a few atoms.
Materials in this class include
graphene \cite{Novoselov2004},
\emph{transition metal dichalcogenides} (\tmds) \cite{Mak2010},
and hexagonal boron nitride \cite{Nagashima1995},
which are the \twod equivalent of a metal, semiconductor, and insulator
respectively \cite{Bharadwaj2015}.
These materials are significant as they allow structures to be built where, not
only is the light itself confined on a subwavelength scale, but the materials
themselves are too.

Two questions on the subject of active nanoplasmonics are considered in this
thesis.
The first concerns \emph{stopped light}.
By controlling the dispersion relation of light in a plasmonic waveguide such
that energy is brought to rest at zero group velocity within a gain medium, can
a regime of lasing be entered without a cavity storing the light in a resonant
mode?

Secondly;
If graphene is optically pumped such that its carriers enter a state of
inversion, are plasmon modes supported and can the inversion compensate for any
material losses and lead to amplification of the plasmons?

Though seemingly disjoint, these questions are related by the approach one takes
to answer them.
In both systems plasmons are coupled to an inverted electronic system in order
to induce the stimulated amplification that is to compensate for material
losses.
Analysis of the amplification requires the exact calculation of the plasmon
dispersion relation in a complex frequency picture, which describes the losses
and gains in time.

The problems differ in key areas too.
For stopped light gain is provided by a
single transition of a \threefive semiconductor modelled as a four-level system,
whereas gain in graphene is provided by a continuum of electron states within
the graphene sheet itself.
This means graphene is host to broadband spontaneous emission, whereas all
emission in stopped light is monochromatic around the lasing frequency.
Finally, Stopped light structures host surface plasmon polaritons, that retain
a degree of photonic character and with the potential of outcoupling.
Graphene on the other hand hosts plasmons, which are the short wavelength limit.
Their character is more Coulombic than photonic, and the material response must
be considered to include nonlocal effects.

The stopped light lasing question is approached by designing a structure
which supports a single-frequency low-dispersion stopped light band.
The band is optimised to support a broad range of wavevectors, allowing
energy to be confined to narrow widths.
Then by examining the frequency domain characteristics of an optimised
structure, the parameters for a gain medium are selected.
Time domain simulations are then conducted to examine the transition from
amplified spontaneous emission to lasing, and to explain the lasing dynamics.

In order to get a handle on the dynamics of plasmons in graphene, the complex
frequency plasmon dispersion relation must be solved for exactly within the
random phase approximation framework.
A formalism is developed for determining polarisabilities of graphene with
nonequilibrium electron distributions.
Plasmon dispersion relations are calculated for graphene in a state of
photoinverted quasiequilibrium.
The formalism is to include the effects of Drude collision loss, finite
temperature and doping by impurities as each of these effects will seek to
reduce the gain available.
Finally a dynamic study of the carrier system relaxation, coupled to spontaneous
and stimulated plasmon and phonon emission is conducted.

The thesis is in five chapters:
The first, this overarching introduction, sets the scope of the two main
questions in a context of active planar nanoplasmonics.
Chapter 2 contains theory of classical electromagnetism in homogeneous media
leading to a frequency domain analysis of waveguides.
The finite difference time domain method is introduced to complement and go
beyond the frequency domain analysis.
Finally the evolutionary algorithm method of optimisation is introduced in the
context of plasmonic waveguides.
Chapters 3 and 4 are the main content chapters, each with their own
introduction, literature, and conclusion, on the topics of
\emph{stopped light lasing} and
\emph{plasmons in nonequilibrium graphene} respectively.
Chapter 5 concludes the thesis in a unifying context and offers an outlook.
