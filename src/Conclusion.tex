\chapter{Conclusion}

This thesis has been themed on controlling light in two dimensions with active
plasmonics.
Two particular topics were focused on, stopped light lasing and gain in
nonequilibrium graphene, where in both plasmons are coupled to electronic
systems to exploit the stimulated emission into plasmonic channels.

The question asked of stopped light was:
By controlling the dispersion relation of light in a plasmonic waveguide such
that energy is brought to rest at zero group velocity within a gain medium, can
a regime of lasing be entered without a cavity storing the light in a resonant
mode?

Designing a structure than not only stops light, but minimises the dispersion
over a large range of wavevectors, allows a wavepacket to be formed that
localises over a gain medium.
The localisation derives from balanced and opposing energy flows in metallic and
dielectric layers that form vortices around the edges of the gain medium.
Lasing is indeed shown to be possible;
time domain simulations show the transition from amplified spontaneous emission
to lasing via characteristic relaxation oscillations.

The dynamics of the lasing mode are unusual.
Energy is localised, but the mode does not form a in standing wave, but rather a
propagating one with zero group velocity.
This is analogous to a barber’s pole, where the stripes are continually moving
but the pole stays in place.
The stopped light laser is a source of coherent plasmons that are well
described with a complex wavevector.
This is in contrast to the lasing mode itself which is described with complex
frequencies.

From here, there are a number of avenues one may take to build on this work.
Firstly, simulations have been done in \twod
(one dimension parallel to the structure, $x$, and the other perpendicular, $z$
).
In the third (homogenised) dimension, $y$, emitters are coherent with
each other and of equal inversion.
In \threed this need not be the case.
Additionally, polarisation effects that were not present in \twod become
available adding extra dynamics that must be studied.

This all should lead to an experimental realisation of a stopped light
structure.
Theoretical models and simulations should be geared to inform the design of a
structure that can be fabricated in the lab, accounting for imperfections and
variations not present in the idealised model.

Turning to graphene, the question posed in the introduction can now be answered:
If graphene is optically pumped such that its carriers enter a state of
inversion, are plasmon modes supported and can the inversion compensate for any
material losses and lead to amplification of the plasmons?

Graphene has been shown to support plasmons with gain, including under realistic
conditions of finite temperature, chemical doping, and collision loss.
In order to arrive at this answer a model for calculating polarisabilities for
nonequilibrium carrier distributions was derived and this allowed for the
calculation of complex-frequency plasmon dispersion relations which encode the
gain and loss.

It has also been shown that the relaxation of hot carrier distributions is
ultra-fast due to spontaneous broadband nonequilibrium emission of plasmons on
a timescale of $\sim100\fs$.
This is in agreement with experimental findings that have thus far relied on
theoretically suppressed Auger processes to justify the observed rates.

To build on this work, one may wish to consider multi-layer graphene or other
\twod materials such as \tmds.
The model for calculating nonequilibrium polarisabilities is not specific to
monolayer graphene, and could be applied to such materials, and indeed to
models of monolayer graphene beyond the Dirac cone approximation.
This could lead to the construction of graphene based plasmonic devices, beyond
suspended graphene, or graphene on a simple substrate that has been considered
here.

Another extension worth considering is anisotropic excitation:
When graphene is photoexcited, carriers pairs are produced with a definite
momentum distribution on the cone.
At present, this can not be modelled in this framework, but should be
investigated in order to consider the dynamics of photoexcitation more fully.

An obvious question that arises from this thesis is can the topics of graphene
and stopped light be combined to form a graphene plasmonic stopped light laser?
The fast inversion relaxation and a broadband spontaneous emission of graphene
make it unsuitable for lasing by itself.
However, this could be mitigated by using graphene as the metallic component of
an \sl structure, not as the emitter, and a \tmd for the semiconductor.
\textsc{Tmd}s have a finite direct bandgap which should make them ideal for such
a purpose.
This would allow for a stopped light laser where, not only is the mode
subwavelength, but the entire device is too.
